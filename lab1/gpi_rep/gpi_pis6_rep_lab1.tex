\documentclass[12pt, a4paper, simple]{eskdtext}

\usepackage{hyperref}
\usepackage{env}
\usepackage{_sty/gpi_lst}
\usepackage{_sty/gpi_toc}
\usepackage{_sty/gpi_t}
\usepackage{_sty/gpi_p}
\usepackage{_sty/gpi_u}

% Код
% \ESKDletter{О}{Л}{Р}
% \def \gpiDocTypeNum {81}
% \def \gpiDocVer {00}
% \def \gpiCode {\ESKDtheLetterI\ESKDtheLetterII\ESKDtheLetterIII.\gpiStudentGroupName\gpiStudentGroupNum.\gpiStudentCard-0\gpiDocNum~\gpiDocTypeNum~\gpiDocVer}

\def \gpiDocTopic {Отчёт лабораторной работы №\gpiDocNum}

% Графа 1 (наименование изделия/документа)
% \ESKDcolumnI {\ESKDfontII \gpiTopic \\ \gpiDocTopic}

% Графа 2 (обозначение документа)
% \ESKDsignature {\gpiCode}

% Графа 9 (наименование или различительный индекс предприятия) задает команда
% \ESKDcolumnIX {\gpiDepartment}

% Графа 11 (фамилии лиц, подписывающих документ) задают команды
% \ESKDcolumnXIfI {\gpiStudentSurname}
% \ESKDcolumnXIfII {\gpiTeacherSurname}
% \ESKDcolumnXIfV {\gpiTeacherSurname}

\begin{document}
    \input{_tex/gpi_rep_titlePage.tex}
    \ESKDstyle{empty}
    \begin{center}
        \textbf{\gpiDocTopic}
    \end{center}

    % = = = = = = = =
    \paragraph{} \textbf{Тема}: <<\gpiTopicRep>>

    \paragraph{} \textbf{Цель}: Познакомиться с предметной областью,
    реализовать минимальный функционал без детального проектирования согласно паттерну <<сценарий транзакции>>.

    \paragraph{} \textbf{Что нужно сделать}:

    Определить предметную область согласно варианту (или предложить свою).
    Реализовать несколько действий в рамках выбранной предметнойьобласти согласно паттерну <<сценарий транзакции>>.
    Не требуется полноценное работоспособное приложение, достаточно пары функций.
    Разработать сценарий транзакция для своей темы.

    \paragraph{} \textbf{Исходный код}: 

    % \lstinputlisting[]
    %     {../gpi_src/docker-compose.yml}

    % \lstinputlisting[]
    %     {../gpi_src/Dockerfile}

    \lstinputlisting[language=sql]
        {../gpi_src/docker-entrypoint-initdb.d/pis6_ab.sql}

    \lstinputlisting[language=php]
        {../gpi_src/var/www/html/index.php}

    \lstinputlisting[language=php]
        {../gpi_src/var/www/html/html/head.php}

    \lstinputlisting[language=php]
        {../gpi_src/var/www/html/scripts/connect.php}
        
    \lstinputlisting[language=php]
            {../gpi_src/var/www/html/scripts/env.php}
        
    \lstinputlisting[language=php]
        {../gpi_src/var/www/html/scripts/create.php}

    \lstinputlisting[language=php]
        {../gpi_src/var/www/html/scripts/read.php}

    \lstinputlisting[language=php]
        {../gpi_src/var/www/html/scripts/delete.php}

    \paragraph{} \textbf{Вывод}:
    Создали сценарий транзакции создание данных,
    создали сценарий транзакций чтения данных,
    создали сценарий транзакций удаления данных.

    Сценарий действия, трактуется как бизнес-транзакция: одна процедура на одно действие бизнес-логики
    Может быть ООП, с подпрограммами, но всё равно - читаемый большой кусок кода.

    Плюсы:
    удобная процедурная модель;
    всем понятно;
    определяет границы транзакции;
    всё в одном.

    Минусы:
    слишком <<всё в одном>>;
    повторение, дублирование кода;
    приложение без отчетливой структуры;
    нормально только при небольшой сложности бизнес-логики

    % = = = = = = = =
    % \newpage
    % \addcontentsline{toc}{section}{Список использованных источников}
    % \section*{Список использованных источников}
    \paragraph{} \textbf{Список использованных источников}:

    \begin{enumerate}
        \item[1.] transaction script сценарий транзакции php - YouTube
        - [Электронный ресурс]
        Режим доступа: \url{https://www.youtube.com/watch?v=7K-2SpMhdyU}.
        Дата~доступа:~26.02.2022.
    \end{enumerate}
    \newpage
\end{document}
